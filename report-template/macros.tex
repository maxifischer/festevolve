%!TEX root = proposal.tex
% Packages
\usepackage[show]{chato-notes}
\usepackage{url}
\usepackage{hyperref}
\usepackage{graphicx}
\usepackage{multirow}
\usepackage{amssymb}
\usepackage{amsmath}
\usepackage{centernot}
% avoid conflicts
\let\proof\relax
\let\endproof\relax
\usepackage{amsthm}
\usepackage{algorithm}
\usepackage{algorithmicx}
\usepackage[noend]{algpseudocode}
\usepackage{enumerate}
\usepackage[dvipsnames]{xcolor}
\usepackage{colortbl}
\usepackage[style=base]{caption}
\usepackage{subcaption}
\usepackage{comment}
\usepackage{balance}
\usepackage{cite} %citations are ordered by number
\usepackage[tikz]{bclogo}
\usepackage[normalem]{ulem}
\usepackage{tikz}
\usepackage{nccmath}
\usepackage{enumitem}
\usepackage{balance}
\usepackage{rotating}
\usepackage{pifont}
\usepackage[margin=2cm]{geometry}
\usepackage{etoolbox} % Patch the bibliography title
\usepackage[noindentafter]{titlesec}

\usetikzlibrary{decorations.markings}
%%
%% The next line says how the "vertex" style of nodes should look: drawn as small circles.
\tikzstyle{vertex}=[circle, draw, inner sep=0pt, minimum size=4pt, fill=black]
\newcommand{\vertex}{\node[vertex]}


\hypersetup{
%    bookmarks=true,         % show bookmarks bar?
    unicode=false,          % non-Latin characters in Acrobats bookmarks
    pdftoolbar=true,        % show Acrobats toolbar?
    pdfmenubar=true,        % show Acrobats menu?
    pdffitwindow=false,     % window fit to page when opened
    pdfstartview={FitH},    % fits the width of the page to the window
    pdftitle={My title},    % title
    pdfauthor={Author},     % author
    pdfsubject={Subject},   % subject of the document
    pdfcreator={Creator},   % creator of the document
    pdfproducer={Producer}, % producer of the document
    pdfnewwindow=true,      % links in new window
    colorlinks=true,       % false: boxed links; true: colored links
    linkcolor=black,          % color of internal links
    citecolor=black,        % color of links to bibliography
    filecolor=black,      % color of file links
    urlcolor=black,           % color of external links
}



\newcommand*{\graphpair}[2]{\langle #1, #2 \rangle}
\newcommand*{\tuple}[1]{(#1_1, ..., #1_n)}
\newcommand*{\range}[3]{#1 = #2 ... #3}
\newcommand*{\pair}[2]{(#1,#2)}
\newcommand*{\triple}[3]{(#1,#2,#3)}
\newcommand*{\edge}[4]{#1 \overset{#3}{#4} #2}
\newcommand*{\uedge}[2]{\edge{#1}{#2}{}{--}}
\newcommand*{\dedge}[2]{\edge{#1}{#2}{}{\rightarrow}}
\newcommand*{\luedge}[3]{\edge{#1}{#2}{#3}{--}}
\newcommand*{\ldedge}[3]{\edge{#1}{#2}{#3}{\rightarrow}}
\newcommand*{\expgname}[1]{#1^+}
\newcommand*{\expgraph}[3]{\expgname{#1}:\graphpair{#2^+}{#3^+}}


%Paper macros



%Style macros
\renewcommand\abstractname{Summary}
%\patchcmd{\thebibliography}{\section*}{\section}{}{}
\renewcommand{\refname}{Bibliography}

%% Work package styles (DO NOT touch)
\renewcommand\theparagraph{\Alph{paragraph}}
\renewcommand\thesubparagraph{[\Alph{paragraph}\arabic{subparagraph}]}
\titleformat{\paragraph}{\normalfont\large\bfseries}{\theparagraph}{1em}{}
\titleformat{\subparagraph}{\normalfont\normalsize\bfseries}{\thesubparagraph}{1em}{}
\titlespacing*{\subparagraph}{0pt}{3.25ex plus 1ex minus .2ex}{1.25ex plus .1ex}
\newcommand{\workpack}{\paragraph}
\newcommand{\subworkpack}{\subparagraph}

\setcounter{secnumdepth}{5}

%Algorithmic macros
\renewcommand{\algorithmicrequire}{\textbf{Input:}}
\renewcommand{\algorithmicensure}{\textbf{Output:}}
\renewcommand{\algorithmicforall}{\textbf{for each}}
\algdef{SE}[DOWHILE]{Do}{DoWhile}{\algorithmicdo}[1]{\algorithmicwhile\ #1}%

%Complexity macros
\newcommand*{\NP}{$\mathbf{NP}$}
\newcommand*{\NPhard}{$\mathbf{NP}$-hard}
\newcommand*{\NPcomplete}{$\mathbf{NP}$-complete}
\newcommand*{\SPcomplete}{$\mathbf{\#P}$-complete}
\newcommand*{\SPhard}{$\mathbf{\#P}$-hard}
\newcommand*{\bigO}{\mathcal{O}}

% Paragraphs
\newcommand{\spara}[1]{\smallskip\noindent{\bf #1}}
\newcommand{\mpara}[1]{\medskip\noindent{\bf #1}}
\newcommand{\para}[1]{\noindent{\bf #1}}

\newcommand{\blackBox}{$\blacksquare$}



\newcommand{\removed}[1]{
  \colorbox{red!30}{{\scriptsize REMOVED}}
  {\color{red}#1}
}

\newcommand{\del}[1]{
  {\color{red}\sout{#1}}
}


\newcommand{\added}[1]{
{\color{ForestGreen}#1}
}